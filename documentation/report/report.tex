\documentclass[a4paper]{article}

\usepackage[utf8]{inputenc}
\usepackage{hyperref}

\usepackage[backend=bibtex]{biblatex}
\addbibresource{references.bib}

\title{Bolognese\\{\large solving the Bologna scheduling problem}}
\author{Alon Dolev \and Joey Ezechiels \and Volker Lanting}

\begin{document}
\maketitle
\begin{abstract}
This document describes the design and implementation process of an application which solves the Bologna scheduling problem.
To do so, it uses an existing constraint programming library for Scala, called OscaR.
In the process of solving the Bologna scheduling problem, a functional interface to OscaR was created.
This interface consists of a JavaScript library to build a constraint model, and a scala web server (using scalatra) which accepts a constraint model and tries to solve it using OscaR.
\end{abstract}

\newpage

%===========   The problem domain   ============

\section{The problem domain}
In this section, a short description of the Bologna scheduling problem will be given.
Then we will further specify the problem that we set out to solve and outline the scope of this project.

	%===========   The Bologna scheduling problem   ============

\subsection{The Bologna scheduling problem}
In the Bologna process, courses (Modules) have a fixed number of points (ECTS) representing the workload required to complete the course.
To complete a study program, a certain amount of ECTS (total required ECTS) has to be obtained by succesfully completing Modules.
There can be more Modules than required to reach the total required ECTS.
In that case students can decide which Modules to take, although they are limited in their choice by some extra constraints.

Each Module should be booked in a Category, but not all Modules can be booked in all Categories.
Also, there should be enough Modules booked in each Category to reach the minimum required amount of ECTS of the Category.
Categories also have a maximum amount of ECTS.
If the ECTS of all Modules booked in a Category exceeds this amount, then the student only gets credit for this maximum amount of ECTS.
The remainder of the ECTS will therefore not count towards the total required ECTS of the study program.

The problem of booking Modules in Categories in such a way that the total required amount of ECTS is reached without violating any of the constraints is what we call the Bologna scheduling problem.

	%===========   The problem domain for this project   ============

\subsection{The problem domain for this project}
In this section, we will first discuss the main entities which define an instance of the Bologna scheduling problem.
Then we will briefly touch upon the extend of the problem domain and what the scope for this project was.

		%===========   The problem domain for this project   ============

\subsubsection{Main problem entities}
\label{sec:problem-parts}
We defined three main parts that the Bologna scheduling problem consists of:
\begin{description}
	\item[Module] \hspace*{1em}
	\begin{itemize}
		\item name -- the name of the Module
		\item ects -- the amount of ECTS for the Module
		\item categories -- the collection of Categories in which this Module is allowed to be booked
	\end{itemize}

	\item[Category] \hspace*{1em}
	\begin{itemize}
		\item name -- the name of the Category
		\item min\_ects -- the minimum amount of ECTS that should be booked in the Category
		\item max\_ects -- the maximum amount of ECTS that a student can get credit for in this Category
	\end{itemize}

	\item[total\_ects] the total amount of creditable ECTS that should be booked and completed in order to pass the study program
\end{description}
Given a collection of Modules, a collection of Categories and a total amount of required ECTS, we can define all required constraints and compute a valid booking of Modules.

		%===========   The problem domain for this project   ============

\subsubsection{Limited scope}
\label{sec:problem-scope}
Although we can describe the general problem with the entities outlined in 
\hyperref[sec:problem-parts]{Section \ref*{sec:problem-parts}}, 
students probably want more than just a `random' schedule, even though completing it would mean completing their study program.
We will now briefly discuss some \emph{optimization goals} the student might have. 
These will be discussed in more detail, along with other extensions of the application, in 
\hyperref[sec:future]{Section \ref*{sec:future}}

For example, modules are usually given only during certain time periods (like quarters or semesters), and a student might want to balance the ECTS load as evenly as possible over all time-periods.
Alternatively, she might want to minimize the time required to finish all booked modules, minimize the amount of ECTS she won't get credit for, or some combination of these optimization goals.

For this project we will focus only on the optimization goal of minimizing the amount of booked ECTS points.
This comes down to minimizing the amount of booked ECTS points for which the student won't get credit (overflowing Category.max\_ects or exceeding total\_ects).

%============   The general approach   ==========

\section{The general approach}
Due to time constraints, the domain of the problem has been restricted as described in
\hyperref[sec:problem-scope]{Section \ref*{sec:problem-scope}}.
We chose to use Constraint Programming to tackle the problem, as this allows the addition of extra constraints later on.
The problem can be rewritten as a constraint model, as described in 
\hyperref[sec:constraint-model]{Section \ref*{sec:constraint-model}}, which can in turn be solved by a constraint solver.

This project was also meant as a Scala learning experience, so we limied the possible choices for constraint solvers to those with a Scala interface.
We compared two possible constraint solvers: OscaR \cite{oscar} and Copris \cite{copris}.

Copris seems to be rather new and lacks English documentation, apart from the rather limited API scaladoc.
It offers a more concise interface to constraint programming than Oscar, but has no built in optimization support like minimize or maximize.
OscaR on the other hand has an English wiki, although with broken links, but there is also a blog with some documented examples \cite{hakank}.
Furthermore, they offer a builtin minimize function.


Based on documentation and the presence of a minimization function which can be used for optimization goals, we went for OscaR.

%============   Design of the application   ==========

\section{Design of the application}
\label{sec:constraint-model}
Given a problem instance as discussed in
\hyperref[sec:problem-parts]{Section \ref*{sec:problem-parts}}, 
we can rewrite the problem into the following constraint model:
\begin{description}
\item[variables]
	For each pair of a Category and Module $(c,m)$ where the Module is allowed to be booked in the Category (i.e. $c\in m.categories$), define a decision variable $d_{cm}\in\{0,1\}$ which will eventually be $1$ if the Module is booked in the Category and $0$ if it is not.

\item[constraints]\hspace*{1em}
\begin{itemize}
	\item
	A Module can only be booked once:  
	\\for each Module $m$: $\sum_{c\in m.categories}d_{cm}\leq 1$

	\item
	A Category $c$ should have at least have enough booked ECTS to reach its minimum amount of required ECTS:
	\\ define $booked_c$ as the amount of ECTS booked for a Category $c$:
	
	$booked_c=\sum_{m\in Modules\wedge c\in m.categories} d_{cm}*m.ects$

	then for each Category $c$: $booked_c \geq c.min\_ects$

	\item the total amount of creditable, booked ECTS should be at least as much as the total required ECTS of the study program:
	\\define $creditable_c$ as the creditable, booked ECTS of a Category $c$:
	
	$creditable_c=min(c.max\_ects, booked_c)$

	then $\sum_{c\in Categories}creditable_c  \geq total\_ects$
\end{itemize}

\item[optimization goal]
	the optimization goal is to minimize the amount of booked ECTS:
	 $minimize(\sum_{c\in Categories}booked_c)$
\end{description}

%===========   Future Research   ==========

\section{Future Research}
\label{sec:future}
\textbf{TODO}

\printbibliography

\end{document}