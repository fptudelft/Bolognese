\documentclass[a4paper]{article}

\usepackage[utf8]{inputenc}
\usepackage{hyperref}

\usepackage[backend=bibtex]{biblatex}
\addbibresource{references.bib}

\title{Bolognese\\{\large solving the Bologna scheduling problem}}
\author{Alon Dolev \and Joey Ezechiels \and Volker Lanting}

\begin{document}
\maketitle
\begin{abstract}
This report was created as part of a project for the course Functional Programming at the Delft University of Technology.
It describes the design and implementation process of an application which solves the Bologna scheduling problem.
To solve the problem, an existing constraint programming library for Scala, called OscaR, 
is used.
In the process of solving the Bologna scheduling problem, a functional interface to OscaR was created.
This interface consists of a a collection of AST node constructors to build a constraint model, 
and an OscaR monad, which is created from the constraint model and encapsulates the side-effecting OscaR solving process.
\end{abstract}

\newpage

%===========   Introduction   ============

\section{Introduction}
This report discusses the design and implementation of an application which solves the Bologna scheduling problem.
In the Bologna scheduling problem a student tries to book his/her courses in categories,
such that he/she will meet all the requirements to pass the study program
when all booked courses are successfully completed.

As this application is created as part of a project for a Functional Programming course, 
its design should incorporate concepts of functional programming.
Therefore, this report will pay special attention to the concepts we incorporated during the design of the application, 
and why we chose to use them.

The secondary goal of this project was, for us, to learn a new programming language
and to get more familiar with applying functional programming concepts in a real application.
We chose Scala as the language for this project, as none of us had used it before.

%===========   The problem domain   ============

\section{The problem domain}
\label{sec:problem-domain}
In this section, a short description of the Bologna scheduling problem will be given.
Then we will further specify the problem that we set out to solve and outline the scope of this project.

	%===========   The Bologna scheduling problem   ============

\subsection{The Bologna scheduling problem}
In the Bologna process, courses (Modules) have a fixed number of points (ECTS) representing the workload required to complete the course.
To complete a study program, a certain amount of ECTS (total required ECTS) has to be obtained by succesfully completing Modules.
There can be more Modules than required to reach the total required ECTS.
In that case students can decide which Modules to take, although they are limited in their choice by some extra constraints.

Each Module should be booked in a Category, but not all Modules can be booked in all Categories.
Also, there should be enough Modules booked in each Category to reach the minimum required amount of ECTS of the Category.
Categories also have a maximum amount of ECTS.
If the ECTS of all Modules booked in a Category exceeds this amount, then the student only gets credit for this maximum amount of ECTS.
The remainder of the ECTS will therefore not count towards the total required ECTS of the study program.

The problem of booking Modules in Categories in such a way that the total required amount of ECTS is reached without violating any of the constraints is what we call the Bologna scheduling problem.

	%===========   The problem domain for this project   ============

\subsection{The problem domain for this project}
We will first discuss the main entities which define an instance of the Bologna scheduling problem.
Then we will briefly touch upon the extend of the problem domain and what the scope for this project was.

		%===========   The problem domain for this project   ============

\subsubsection{Main problem entities}
\label{sec:problem-parts}
We defined three main parts that the Bologna scheduling problem consists of:
\begin{description}
	\item[Module] \hspace*{1em}
	\begin{itemize}
		\item name -- the name of the Module
		\item ects -- the amount of ECTS for the Module
		\item categories -- the collection of Categories in which this Module is allowed to be booked
	\end{itemize}

	\item[Category] \hspace*{1em}
	\begin{itemize}
		\item name -- the name of the Category
		\item min\_ects -- the minimum amount of ECTS that should be booked in the Category
		\item max\_ects -- the maximum amount of ECTS that a student can get credit for in this Category
	\end{itemize}

	\item[total\_ects] the total amount of creditable ECTS that should be booked and completed in order to pass the study program
\end{description}
Given a collection of Modules, a collection of Categories and a total amount of required ECTS, we can define all required constraints and compute a valid booking of Modules.

		%===========   The problem domain for this project   ============

\subsubsection{Limited scope}
\label{sec:problem-scope}
Although we can describe the general problem with the entities outlined in 
\hyperref[sec:problem-parts]{Section \ref*{sec:problem-parts}}, 
students probably want more than just a `random' schedule, even though completing it would mean completing their study program.
We will now briefly discuss some \emph{optimization goals} the student might have. 

Modules are usually given only during certain time periods (like quarters or semesters),
and a student might want to balance the ECTS load as evenly as possible over all time-periods.
Alternatively, she might want to minimize the time required to finish all booked modules, 
minimize the amount of ECTS she won't get credit for, 
or some combination of these optimization goals.

For this project we will focus only on the optimization goal of minimizing the amount of booked ECTS points.
This comes down to minimizing the amount of booked ECTS points for which the student won't get credit (overflowing Category.max\_ects or exceeding total\_ects).




%============   Design of the application   ==========

\section{Design of the application}
\label{sec:design}
This section will describe the design process of the application.
It will pay specific attention to the functional programming concepts that we
chose to use.

First, the general approach we decided to take (constraint programming) is discussed,
as well as why it was chosen.
Then, the application will be broken down into the steps required to solve our problem.
For these steps we will briefly discuss whether they are suited for the application
of functional programming concepts.
Finally, we will outline the design of the two main parts of the application:
the user interface (UI) and the server.

	%============   The general approach   ==========

\subsection{The general approach: constraint programming}
\label{sec:general-approach}
Due to time constraints, the domain of the problem has been restricted as described in
\hyperref[sec:problem-scope]{Section \ref*{sec:problem-scope}}.
We chose to use Constraint Programming to tackle the problem, 
as this allows the addition of extra constraints later on.
Ensuring that we can easily extend the application after the first version is complete.

The problem can be rewritten as a constraint model, as described in 
\hyperref[sec:constraint-model]{Section \ref*{sec:constraint-model}}, which can in turn be solved by a constraint solver.

		%============   Picking a constraint solver   ==========

\subsubsection{Picking a constraint solver}
As this project was also meant as a Scala learning experience, 
we limited the possible choices for constraint solvers to those with a Scala interface.
We compared two such constraint solvers: OscaR \cite{oscar} and Copris \cite{copris}.

Copris seems to be rather new and lacks English documentation, apart from the rather limited API scaladoc.
It offers a more concise interface to constraint programming than Oscar, but has no built in optimization support like minimize or maximize.
OscaR on the other hand has an English wiki, although with broken links, but there is also a blog with some documented examples \cite{hakank}.
Furthermore, they offer a builtin minimize function.

Based on documentation and the presence of a minimization function which can be used for optimization goals, we went for OscaR.

		%============   The problem as constraint model   ==========

\subsubsection{The problem as constraint model}
\label{sec:constraint-model}
Given a problem instance as discussed in
\hyperref[sec:problem-parts]{Section \ref*{sec:problem-parts}}, 
we can rewrite the problem into the following constraint model:
\begin{description}
\item[variables]
	For each pair of a Category and Module $(c,m)$ where the Module is allowed to be booked in the Category (i.e. $c\in m.categories$), define a decision variable $d_{cm}\in\{0,1\}$ which will eventually be $1$ if the Module is booked in the Category and $0$ if it is not.

\item[constraints]\hspace*{1em}
\begin{itemize}
	\item
	A Module can only be booked once:  
	\\for each Module $m$: $\sum_{c\in m.categories}d_{cm}\leq 1$

	\item
	A Category $c$ should have at least have enough booked ECTS to reach its minimum amount of required ECTS:
	\\ define $booked_c$ as the amount of ECTS booked for a Category $c$:
	
	$booked_c=\sum_{m\in Modules\wedge c\in m.categories} d_{cm}*m.ects$

	then for each Category $c$: $booked_c \geq c.min\_ects$

	\item the total amount of creditable, booked ECTS should be at least as much as the total required ECTS of the study program:
	\\define $creditable_c$ as the creditable, booked ECTS of a Category $c$:
	
	$creditable_c=min(c.max\_ects, booked_c)$

	then $\sum_{c\in Categories}creditable_c  \geq total\_ects$
\end{itemize}

\item[optimization goal]
	the optimization goal is to minimize the amount of booked ECTS:
	 $minimize(\sum_{c\in Categories}booked_c)$
\end{description}

	%============   Breakdown of the system   ==========

\subsection{Breakdown of the application}
\label{sec:breakdown}
Here we will break down the steps required for our application and
try to indentify the places where functional programming concepts can really shine.
Then, we will discuss the design of each part of the system in turn.

The application will consist of the following steps:
\begin{enumerate}
	\item Web user interface (UI) to ask the user to create modules and categories
	and indicate where the modules can be booked.
	\item Gathering the user supplied information into some data format.
	\item Passing this data to the server.
	\item Transforming the data into some kind of format OscaR understands.
	\item Running OscaR.
	\item Mapping the results back to a user readable data format.
	\item Passing the solution to the client.
	\item Displaying the solution.
\end{enumerate}
The UI will involve IO, so it will inherently be stateful.
This means it will not be very easy to apply functional programming concepts to the UI.
The user interface will be discussed in more detail in 
\hyperref[sec:design-ui]{Section~\ref*{sec:design-ui}}.

Data transformation on the other hand is where functional programming shines,
allowing for very concise notation and no side-effects.
Data transmission will be taken care of by some libraries (jQuery and scalatra), 
so we will not discuss it any further.

This only leaves running OscaR, which has an imperative interface, 
where constraint variables and constraints should be added to a CPSolver object.
The Solver is mutated, and every time you add a constraint 
the involved constraint variables are updated.
Obviously, this means there are side-effects to running OscaR.
Therefore, the OscaR operations should be encapsulated and the functional style data
transformation should be shielded from the side effects.
A monad is the obvious choice for this situation.


	%===========   Design of the UI   ==========
\subsection{The User Interface}
\label{sec:design-ui}
The user interface will be a web application, 
where the user can create a problem instance and ask the server to solve it.
Due to the statefull nature of IO, and therefore of user interfaces,
it was not easy to apply functional programming concepts to the user interface.

First, we will discuss the creation of a problem instance, 
and how we planned on adding a functional flavour to it.
Then, the process of obtaining the data that the user created and sending
it to the server is discussed.
This part consists of obtaining data and transforming it, 
allowing us to use concepts like Functors for easy transformation.

		%===========  Enabling user-generated content   ==========
\subsubsection{Enabling user-generated content}
To enable the user to create a problem instance of the Bologna scheduling problem
the user inteface needs to be able to do the following things:
\begin{itemize}
	\item adding Categories and Modules
	\item displaying Categories and Modules
	\item validating user input
\end{itemize}
Since it is a web interface, we will use the `native languages of the web' HTML, CSS,
and JavaScript.
We decided to try something new for the user interface, 
to maximize our learning experience.
Therefore, we chose to use a reactive programming approach with the Flapjax library \cite{flapjax} to allow the creation, validation and display of Categories and Modules.

\subsubsection{Obtaining data for transmission}
\label{sec:ObtainingData}
Solving the actual problem is done on the server, so the
user-generated problem instance has to be transmitted. To obtain the
data of the problem instance, we process the displayed HTML tables.\\

While this approach would not be advisable for production systems
(there a single data model that propagates its data both to the HTML
and to other parts of the logic), given the scope and time constraints
of this project it is an acceptable and technically simple tradeoff.\\

There are 2 kinds of HTML table in this context:
\begin{itemize}
\item for Categories
\item for Modules
\end{itemize}

Each table is marked in a specific manner (more about this in section
\ref{sec:ProcessingClientData}), and the client side uses these
markings are to extract relevant information.
One single mechanism (called the \textit{mappings mechanism} by its
author as that is what is happening conceptually) is used by the
client side Javascript code to retrieve information for both
categories and modules, and it is flexible enough to adapt to changes
to the column layout in either table without requiring code changes.\\

To achieve this flexibility, the mappings mechanism extracts both the
column layout (which includes the column names, the column ordering
and the number of columns and is formed by the HTML table header) and
the information contained in the HTML table data rows and creates,
per data row, a mapping (hence the name) from the header layout to
that particular instance described by the data row. Hence each row
results in a Javascript map.\\

All the mappings are then collected and transformed into JSON before
being sent to the server using a POST request.


		%===========   Design of the server   ==========
\subsection{The server-side}
\label{sec:design-server}
The server will obtain the Bologna scheduling problem instance data from the client
and will have to build a constraint model from that, on which OscaR can be run.
Then the result will be mapped to a Bologna format again and is sent to the client.

First, our method of obtaining and processing data from the client is discussed.
Then, a stateless, side-effect free constraint model for OscaR is designed.
Finally, we will discuss the design of the OscaR monad, 
which encapsulates the side-effecting OscaR operations.

\subsubsection{Obtaining data from the client}

Upon receiving a POST request to the ``/'' URI, the server
proceeds to transform the data contained in the POST variables to a
Scala object structurally very similar to the JSON object sent by the
client side Javascript code (see section
\ref{sec:ObtainingData}). This is all done by Scalatra framework code
and takes almost no effort on the part of the programmer, save for
defining a Scala case class (in our case calles JsonData) for
deserialization purposes.\\

Once this step has been completed, the data contained within the
object is harvested and transformed to a format the OscaR monad can
understand.

\subsubsection{The side-effect free constraint model}
Since the current constraint model of OscaR has side effects,
a new constraint model should be created.

This model will contain the following elements:
\begin{description}
	\item[variables] integer range variables.
	This is the stateless counterpart of the OscaR CPVarInt.
	\item[constraints] constraints which limit the possible values of the variables.
	This limiting should only be done when the OscaR monad is executed,
	as it includes side-effects.
	\item[optimization goals] an optimization goal like minimizing some function.
\end{description}
These elements will be constructed as an abstract syntax tree (AST).
This means that constructing them is like chaining functions (AST node constructors). 
It also means that we can transform it to OscaR format with something similar to fold algebras.

\subsubsection{The OscaR monad}
From the constraint model that we just discussed, we can create an OscaR monad of type 
\verb|OscaR[Collection[CPMFixedVar]]|, 
where a CPMFixedVar is a variable from the constraint model, 
which got fixed to a specific value by running OscaR.

Since the OscaR monad encapsulates the side effects,
we can transform this collection of fixed variables to something closer to the
data format of the Bologna scheduling problem (e.g. Categories and Modules),
completely without side effects.
Only when we get the value out and send it to the client will the side effects occur.

A monad in Haskell should have a function \verb|return : a->m a| and a function
 \verb|>>= : m a->(a->m b)->m b|, often pronounced as `bind.'
In Scala, flatMap (a different name for bind) and 
map (often referred to as fmap since monads are functors) should be defined.
As this project is about using functional programming concepts, 
we will define all four `standard' monad functions for the OscaR monad:
\begin{verbatim}
unit/return : a -> OscaR a
map/fmap	: OscaR a -> (a -> b) -> OscaR b
flatten/join: OscaR (OscaR a) -> OscaR a
flatMap/bind/>>= : OscaR a -> (a -> OscaR b) -> OscaR b
\end{verbatim}

%===========   The implementation process   ==========
\section{The implementation process}
\label{sec:impl}
{\Large TODO:} create structure and discuss stuff

\subsection{User Interface}
\subsubsection{Flapjax}
\label{sec:impl-ui-alon}
The user can create Categories and Modules.
When the user creates a new Category or Module it is displayed
as a row in an HTML table.
Once the user is ready and wants to compute a schedule,
the data is read from these tables and sent to the server.
For the creation of such table rows, displaying them and validating 
the user input, we used the Flapjax library.
We will now discuss how the implementation process using 
the Flapjax library went.

Adding a row of data to a table is handled with input boxes. We extract the
values using a "behavior" or "signal". We then use special "liftB" functions
, which are provided by the library, to apply validation functions to the
streams and thus extract booleans. Each input is validated separately and then,
using the "andB" function, we determine whether the complete input-row is valid.

We display this validation result using a set of "insertValueB" functions.
The functions are dependent on the validated streams mentioned above. This
means they are executed on changes to the valid state of the extracted input.
As a result they switch two values for HTML properties (such as border-color)
directly on DOM-elements and choose depending on the valid-state they are
dependent on.

Functional Reactive Programming, and Flapjax specifically, proved to be useful
for achieving the validation tasks mentioned so far. 
Adding a new row to the table on the
other hand was challenging and we do not consider our solution elegant.

The first problem we had to solve is that our "behaviors" for the input fields
are all independent from each other. At what point should we add the new row?
We cannot react to the full-row-validity because we need to keep in mind that
the user might not want to add every possible valid input combination. Further,
what if they want to add two equal rows? The solution is to introduce an "event",
bound to a submit button. The button is hidden and shown depending on the
validation state of the row using the same "inputValueB" mechanism mentioned
above. This simple technique ensures that no invalid data will be added by
the user.

The submit button event is coupled with a series of "snapshot" functions which
extract a value at the time of the event and from the input signals. Here we
encounter the problem that our data is tabular and every extracted value needs
to be at the right place in the table. The collection of all input data does not
follow this ordering per se. We solved this by introducing specific delays for
each input which just means that we "sample" the input streams into the data
format we need. The "sampled" values are then combined into a single, "serialized
and ordered" event using a "merge" statement.

The next problem arises quickly: Given a continuous stream of serialized and
ordered "sampled" input values, how can we group them into HTML table rows?

The main difficulty is that HTML input and output tags have to be matched. We
thus would like to have some function which extracts n-samples and packs them
into a "$<$tr$>$"-tag, giving each corresponding input value its own "$<$td$>$"-tag.
It is clear that we need to use some form of fold function, or in the terms of
Flapjax "collect". Unfortunately, this showed to be very hard to achieve using
Flapjax. The final solution looks like this:
{\footnotesize\begin{verbatim}
  var categoriesCollectedE = categoriesAllE.collectE([], function(item, rest) {
         if(item === -1) {
           var a = rest.pop();
           var b = rest.pop();
           var c = rest.pop();
           rest.push(TR(c, b, a));
         } else {
           rest.push(TD(item));
        }
        return rest;
  }).mapE(function(a) {
    return TABLE(TR(TH("Name"), TH("Minimum Points"), TH("Maximum Points")), a);
  });
\end{verbatim}}
We exploit the flexibility of JavaScript and the way Flapjax implements collect
and simply "pop" three values from the array and push them back as HTML row.
This construct looks a lot like a stack-machine with only one instruction: -1.
The value was additionally injected into the merge input streams from a constant
behavior source. It is introduced because we need some mechanism for adding the
required "$<$td$>$" tags as well. We chose "-1" as instruction because negative
numbers are not used in our setting (as long as there is no new Bologna reform...)

Finally, we map the result into a full table. This was needed, because Flapjax
automatically wraps "unsafe" or "incomplete" HTML structures, such as table
rows, in "$<$span$>$"-tags. By constructing the whole table from scratch on every update
we avoid having spans in the tables which do not render correctly. The efficiency
of such an approach is questionable.

Finally, the same is done again for the Modules. The "same" code is present twice.
This should be tackled in the future.

As we have already mentioned: We are not completely happy with our results.
The concept of Functional Reactive Programming must contain some more tricks we
have not yet discovered in order to solve said problems. One approach which could
be taken for the next time, is to look at algorithms and "patterns" used in hardware
design. Actual hardware can be easily simulated using such "data-flow" concept
and we might use some of those ideas here as well.


\subsubsection{Underscore.js}

Underscore.js is a Javascript library mainly intended to facilitate
functional programming style in Javascript. To that end it provides a
relatively elaborate set of functions, among which are more
``standard'' functional-style functions such as map, reduce (foldl),
reduceRight (foldr), zip, filter etc. While Underscore.js provides a
lot more functionality than just these functions, it is mainly used
because of the functionality named above.\\

Both the setup and the usage are extremely easy and in the experience
of the authors the library works without so much as a single
hitch. Both setup and installation are strongly reminiscent of jQuery,
except that instead of \textbf{\$}.$<$function\_name$>$,
\textbf{\_}.$<$function\_name$>$ is used. Needless to say, the
satisfaction about this library is extremely high.\\

\subsection{The server-side}
\subsubsection{Processing the data from the client}
\label{sec:ProcessingClientData}
{\Large TODO:} Joey

\subsubsection{Creating a constraint model}
We constructed a AST-like constraint model by creating 
abstract and case classes. 
The abstract classes represent types of AST-nodes, 
while the case classes represent the AST-node constructors.
We tried to keep the constraint model generic enough,
so that others can also define their problems in terms of this model.

Since OscaR really only offers integer variables 
(CPVarBool is just a convenience class which extends CPVarInt),
we only support integer variables in our model (see CPMVarInt).
All variables should be uniquely named, 
so they can be referenced (CPMIntRef) in constraints (AbstractCPMConstraint) later on.
A CPMVarInt with values ranging from 1 to 15 (inclusive), 
can then be constructed with \verb|CPMVarInt("name", 1, 15)|.
We could use it in a constraint like 
\verb|CPMIntLt(CPMIntRef("name"), CPMIntCons(7))|,
which means that the variable should be strictly less than the integer constant 7.

Currently we only have simple integer constraints
like `less than' and `greater than or equals'.
More can be added by extending AbstractCPMConstraint.
These integer constraints can be built from one or more integer 
(or CPMVarInt) typed expressions that extend AbstractCPMIntExp.

Finally we have the AbstractCPMGoal, 
which represents an optimization goal.
We only implemented the bare minimum we needed,
so the only available goal at the moment is CPMMinimize.
This signals that you want to minimize a CPMVarInt typed
expression.
For example, to find a solution which minimizes $5\cdot name$ we use the goal
\\\verb|CPMMinimize(CPMIntMul(CPMIntCons(5), CPMIntRef("name")))|.

All AST-node constructors can be found in ConstraintModel.scala.

\hspace*{1em}\\
To process such a model into OscaR form, 
we created the oscarize function, which is located in OscaR.scala.
Given our time constraints, 
this oscarize function was the closest we could get to the idea of fold algebras.

Running the oscarize function on CPMVarInts returns CPVarInts,
which are used by OscaR during the exploration phase.

Constraints in OscaR are added in a constraints block which should be of type
\verb|=>Unit|.
This allows us to just use a foreach call on the collection of CPMConstraints.
In this foreach call, the constraints will be oscarized.
The oscarize function should add the constraint represented by the CPMConstraint
to the CPSolver object.
To do this, we recursively call oscarize on the CPMIntExps
that this constraint consists of.

If you want to extend our constraint model,
you only have to add a case class and an entry in the correct oscarize function.


\subsubsection{Creating and using the OscaR monad}
{\Large TODO:} Volker


%===========   Conclusion   ==========

\section{Conclusion}
\label{sec:conclusion}
Our application does what is should do:
it solves instances of the Bologna scheduling problem as we specified.
However, it is not going to win a beauty pageant or attract many users in its 
current form.

Our learning goals for this project were definitely met though,
so we will discuss our findings related to the functional programming concepts,
programming languages and libraries we used.
We did not have the time to really master any of these techniques we used,
but we got a feel for what worked for us and what did not.

\subsection{Reactive programming in Flapjax}
As we have already mentioned in 
\hyperref[sec:impl-ui-alon]{Section~\ref*{sec:impl-ui-alon}}:
We are not completely happy with our results.
The concept of Functional Reactive Programming must contain some more tricks we
have not yet discovered in order to solve said problems. 

One approach which could
be taken for the next time, is to look at algorithms and "patterns" used in hardware
design. Actual hardware can be easily simulated using such "data-flow" concept
and we might use some of those ideas here as well.

\subsection{Functional JavaScript with Underscore js}
{\Large TODO:} Joey

\subsection{Monads and Scala}
As far as Scala goes, 
we learned that it is not as concise as Haskell,
and for purely functional programming we would rather use Haskell.

We admit that we only scratched the surface of what Scala has to offer,
but while creating the OscaR monad we had to do quite some juggling with classes.
We hoped to be able to just define a few functions and be done, 
but even though we made our monad extend the Function class,
a function of the same type can still not be seen as an instance of the OscaR monad.
This meant that we could not really think of our monad as a function,
so we could not let the types do the talking (i.e. \verb|OscaR a = ()->a|),
which made it a lot harder to express the correct behaviour.

We will probably not use Scala as a functional language in the future.

\printbibliography

\end{document}
