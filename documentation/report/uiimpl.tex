\label{sec:impl-ui-alon}
The user can create Categories and Modules.
When the user creates a new Category or Module it is displayed
as a row in an HTML table.
Once the user is ready and wants to compute a schedule,
the data is read from these tables and sent to the server.
For the creation of such table rows, displaying them and validating 
the user input, we used the Flapjax library.
We will now discuss how the implementation process using 
the Flapjax library went.

Adding a row of data to a table is handled with input boxes. We extract the
values using a "behavior" or "signal". We then use special "liftB" functions
, which are provided by the library, to apply validation functions to the
streams and thus extract booleans. Each input is validated separately and then,
using the "andB" function, we determine whether the complete input-row is valid.

We display this validation result using a set of "insertValueB" functions.
The functions are dependent on the validated streams mentioned above. This
means they are executed on changes to the valid state of the extracted input.
As a result they switch two values for HTML properties (such as border-color)
directly on DOM-elements and choose depending on the valid-state they are
dependent on.

Functional Reactive Programming, and Flapjax specifically, proved to be useful
for achieving the validation tasks mentioned so far. 
Adding a new row to the table on the
other hand was challenging and we do not consider our solution elegant.

The first problem we had to solve is that our "behaviors" for the input fields
are all independent from each other. At what point should we add the new row?
We cannot react to the full-row-validity because we need to keep in mind that
the user might not want to add every possible valid input combination. Further,
what if they want to add two equal rows? The solution is to introduce an "event",
bound to a submit button. The button is hidden and shown depending on the
validation state of the row using the same "inputValueB" mechanism mentioned
above. This simple technique ensures that no invalid data will be added by
the user.

The submit button event is coupled with a series of "snapshot" functions which
extract a value at the time of the event and from the input signals. Here we
encounter the problem that our data is tabular and every extracted value needs
to be at the right place in the table. The collection of all input data does not
follow this ordering per se. We solved this by introducing specific delays for
each input which just means that we "sample" the input streams into the data
format we need. The "sampled" values are then combined into a single, "serialized
and ordered" event using a "merge" statement.

The next problem arises quickly: Given a continuous stream of serialized and
ordered "sampled" input values, how can we group them into HTML table rows?

The main difficulty is that HTML input and output tags have to be matched. We
thus would like to have some function which extracts n-samples and packs them
into a "$<$tr$>$"-tag, giving each corresponding input value its own "$<$td$>$"-tag.
It is clear that we need to use some form of fold function, or in the terms of
Flapjax "collect". Unfortunately, this showed to be very hard to achieve using
Flapjax. The final solution looks like this:
{\footnotesize\begin{verbatim}
  var categoriesCollectedE = categoriesAllE.collectE([], function(item, rest) {
         if(item === -1) {
           var a = rest.pop();
           var b = rest.pop();
           var c = rest.pop();
           rest.push(TR(c, b, a));
         } else {
           rest.push(TD(item));
        }
        return rest;
  }).mapE(function(a) {
    return TABLE(TR(TH("Name"), TH("Minimum Points"), TH("Maximum Points")), a);
  });
\end{verbatim}}
We exploit the flexibility of JavaScript and the way Flapjax implements collect
and simply "pop" three values from the array and push them back as HTML row.
This construct looks a lot like a stack-machine with only one instruction: -1.
The value was additionally injected into the merge input streams from a constant
behavior source. It is introduced because we need some mechanism for adding the
required "$<$td$>$" tags as well. We chose "-1" as instruction because negative
numbers are not used in our setting (as long as there is no new Bologna reform...)

Finally, we map the result into a full table. This was needed, because Flapjax
automatically wraps "unsafe" or "incomplete" HTML structures, such as table
rows, in "$<$span$>$"-tags. By constructing the whole table from scratch on every update
we avoid having spans in the tables which do not render correctly. The efficiency
of such an approach is questionable.

Finally, the same is done again for the Modules. The "same" code is present twice.
This should be tackled in the future.

As we have already mentioned: We are not completely happy with our results.
The concept of Functional Reactive Programming must contain some more tricks we
have not yet discovered in order to solve said problems. One approach which could
be taken for the next time, is to look at algorithms and "patterns" used in hardware
design. Actual hardware can be easily simulated using such "data-flow" concept
and we might use some of those ideas here as well.
